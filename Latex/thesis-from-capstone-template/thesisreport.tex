\documentclass[11pt]{article}

\usepackage{sectsty}
\usepackage{graphicx}

% Margins
\topmargin=-0.45in
\evensidemargin=0in
\oddsidemargin=0in
\textwidth=6.5in
\textheight=9.0in
\headsep=0.25in

\title{ MIE479 - Literature Review}
\author{ Anojan Palarajah, Bill Tang, Seyon Vasantharajan, Hamed Zakeri }
\date{\today}
\par
\begin{document}

\maketitle	
\pagebreak

% Optional TOC
\tableofcontents
\pagebreak

%--Paper--------------------------------------------------------------------------------------------

\section{Abstract}


\pagebreak
%-------------------------------------------------------------------------------
%---------------------------------------NEW SECTION-----------------------------
%-------------------------------------------------------------------------------

\section{Black-Litterman Papers}

%----------------------------------New Subsection----------------------------
\subsection{\textnormal{}} 

\textbf{Summary: } 
\\*
\\
\textbf{Relevance: } 
%----------------------------------New Subsection----------------------------
%-------------------------------------------------------------------------------
%---------------------------------------NEW SECTION-----------------------------
%-------------------------------------------------------------------------------

\section{Unsorted}

%----------------------------------New Subsection----------------------------
\subsection{\textnormal{Bayesian Portfolio Selection - (Polson, Tew)}} 

\textbf{Summary: } 
\\*
\\
\textbf{Relevance: } 
%----------------------------------New Subsection----------------------------
\subsection{\textnormal{Black-Litterman Model - Mathematical and behavioural finance approaches - (Mankert)}} 

\textbf{Summary: } 
\\*
\\
\textbf{Relevance: } 

%----------------------------------New Subsection----------------------------
\subsection{\textnormal{Global Asset Allocation with Equities, Bonds and Currencies - (Black , Litterman)}} 

\textbf{Summary: } 
\\*
\\
\textbf{Relevance: } 

%----------------------------------New Subsection----------------------------
\subsection{\textnormal{The Black-Litterman Model: Hype or Improvement? - (Salomons)}} 

\textbf{Summary: } 
\\*
\\
\textbf{Relevance: } 

%----------------------------------New Subsection----------------------------
\subsection{\textnormal{Examples of the use of Data Mining in Financial Applications - (Langdell)}} 

\textbf{Summary: } 
\\*
\\
\textbf{Relevance: } 

%----------------------------------New Subsection----------------------------
\subsection{\textnormal{Predicting Stock Prices using Data Mining Techniques - (RadaiDeh, Assaf, Alnagi)}} 

\textbf{Summary: } 
\\*
\\
\textbf{Relevance: } 

%----------------------------------New Subsection----------------------------
\subsection{\textnormal{Review: Application of Data Mining Tools for Stock Market - (Mahajan, Kulkarni0}} 

\textbf{Summary: } 
\\*
\\
\textbf{Relevance: } 

%----------------------------------New Subsection----------------------------
\subsection{\textnormal{}} 

\textbf{Summary: } 
\\*
\\
\textbf{Relevance: } 

%----------------------------------New Subsection----------------------------
\subsection{\textnormal{}} 

\textbf{Summary: } 
\\*
\\
\textbf{Relevance: } 



 
 
 
 
 
 


\pagebreak
%-------------------------------------------------------------------------------
%---------------------------------------NEW SECTION-----------------------------
%-------------------------------------------------------------------------------

\section{Use of Machine Learning in Finance}

%----------------------------------New Subsection----------------------------
\subsection{\textnormal{A Novel Algorithmic Trading Framework Applying Evolution and Machine Learning
for Portfolio Optimization}}

\textbf{Summary: }
\\*
\\
\textbf{Relevance: }

%----------------------------------New Subsection----------------------------
\subsection{\textnormal{The Elements of Statistical Learning (Hastie, Tibishirani,Friedman}}

\textbf{Summary: }
\\*
\\
\textbf{Relevance: }

%----------------------------------New Subsection----------------------------
\subsection{\textnormal{ Identifying Winning Stocks (Balachandran, Saraph, Ang)}}

\textbf{Summary: }
\\*
\\
\textbf{Relevance: }


%----------------------------------New Subsection----------------------------
\subsection{\textnormal{Predicting Short-Term Stock returns - (Lochmiller, Chen)}}

\textbf{Summary: }
\\*
\\
\textbf{Relevance: } 

%----------------------------------New Subsection----------------------------
\subsection{\textnormal{Supervised classification-based stock prediction and portfolio optimization - (Aruk, Eryilmaz, Goldberg)}}

\textbf{Summary: }
\\*
\\
\textbf{Relevance: }

\pagebreak
%----------------------------------New Subsection----------------------------
\subsection{\textnormal{Trend Following Trading under a Regime-Switching Model - (Dai, Zhang, Zhu)}}

\textbf{Summary: }
\\*
\\
\textbf{Relevance: }

\pagebreak
%----------------------------------New Subsection----------------------------
\subsection{\textnormal{Assessing Opinion Mining in Trading Stocks - (Nagappan, Dasu)}}

\textbf{Summary: }
\\*
\\
\textbf{Relevance: }

\pagebreak
%-------------------------------------------------------------------------------
%---------------------------------------NEW SECTION-----------------------------
%-------------------------------------------------------------------------------

\section{Modification Methods to Existing Frameworks}
%----------------------------------New Subsection----------------------------

%----------------------------------New Subsection----------------------------
\subsection{\textnormal{Examining Portfolio Optimization as a Regression Problem - (Brides)}}

\textbf{Summary: }This paper examines different types of mean-variance analysis as a series of related regression models. It treats the risk-free return as the dependent variable and the assets as the independent variables. The coefficients from regression are portfolio weights and the intention is that every portfolio along the efficient frontier will be a solution to the regression problem. Various models are constructed including ones with and without the existence of a risk-free asset. It uses F-tests to evaluate portfolios to test for structural changes in efficient portfolios and uses weighted least squares to deal with volatility clustering, which is the tendency of large absolute movements in asset prices to be followed by more large magnitude changes. 
\\*
\\
\textbf{Relevance: }The usefulness of this paper is in its highly mathematical nature and its creativity. It attempts to recast the mean-variance portfolio problem as a linear regression one and apply analytical tools that become available as a result of it becoming a regression problem. It teaches how to use the regression tool box and conduct regression inference. It also discusses the identification of historical outliers that are influential to final estimates, which can then be disregarded after identification. Overall, this paper serves as a fundamental reference in the case that we decide to build a regression model to deal with the asset allocation problem and hedge inflation.



%----------------------------------New Subsection----------------------------
\subsection{\textnormal{A Step-by-Step Guide to the Black-Litterman Model: incorporating user-specified confidence levels (Idzorek)}}

\textbf{Summary: }This paper introduces the fundamentals of the Black-Litterman Model from reverse optimization to calculating the new Combined Return Vector and discussion on the Tau parameter. It discusses the representation of views 
and their corresponding returns and variances in the various matrices of the Black-Litterman model. Idzorek then introduces his new method for incorporating User-Specified Confidence Levels. He emphasize that determining the uncertainty of the views is one of most difficult parts of the model as it forces users to specify a probability density function for each view. His new method allows users to choose the parameter more intuitively using a 0\% to 100\% scale. The process for deriving the elements of the uncertainty matrix and the corresponding tilts and final portfolio weights are then discussed.
\\*
\\
\textbf{Relevance: }The usefulness of this paper derives from how it takes an existing model and tries to innovate it through the concept of User-Specified Confidence levels of views. This makes it more accessible to non-institutional investors, which may be a goal in the construction of our portfolio framework. The ability to simplify the need to specify entire probability distributions for each view to a simple confidence level is one that we could perhaps utilize as a parameter in our own model.
%----------------------------------New Subsection----------------------------
\subsection{\textnormal{Enhancing the Black-Litterman and Related Approaches:
Views and Stress-test on Risk Factors - (Meucci)}}

\textbf{Summary: }This paper expands the standard Black Litterman Model from MIE377 to incorporate views more general than returns on assets. The paper mathematically derives a method by which to incorporate views on any influential factor so long as is normally distributed.  It gives the mathematical framework, as well as the code required for potential applications.
\\*
\\
\textbf{Relevance: }It gives us a framework that we are already familiar with and a method by which to directly incorporate inflation, if we can convince ourselves that inflation is normally distributed. This however, remains to be seen, as preliminary research indicates that the inflation forecast distributions are not so clear cut.


\pagebreak





\end{document}